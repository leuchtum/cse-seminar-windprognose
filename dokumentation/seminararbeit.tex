% CSE Seminararbeit
% Thema: Ein auf Neuronalen Netzen basierendes Ensemble-Modell zur Windprognose
% Autoren: Alicia Pirwass, Daniel Müller

\documentclass[
12pt, %Schriftgröße
toc=listofnumbered, %Tab.- & Abb.verzeichnis ins TOC
toc=chapterentrydotfill, %TOC: Punkte auch nach Kapitel
numbers=noenddot, %Kapitelüberschrift: Kein Endpunkt z.B. 2.2. --> 2.2
captions=tableheading, %Mehr Platz bei Captionüberschriften (Tabelle)
bibliography=numbered
]{scrreprt}


%%%%% SCHRIFTSATZ, SPRACHE, SCHRIFTART
%%% USE WITH pdflatex, siehe https://tex.stackexchange.com/a/44701
%\usepackage[T1]{fontenc}
%\usepackage[utf8]{inputenc}
%%% USE WITH xelatex
\usepackage{fontspec}
\defaultfontfeatures{Ligatures=TeX}
%%% USE WITH lualatex
%\usepackage{luatextra}
%\defaultfontfeatures{Ligatures=TeX}
%%%
\usepackage[ngerman]{babel}
% Nur mit LuaTex Interpreter nutzbar
%\usepackage{mathfont} % Dieses Packet läd auch fontspec
%\setmainfont{Segoe Pro} % Textschrift separat setzen


%%%%% INHALTSVERZEICHNIS
%\setuptoc{toc}{numbered} % TOC ins TOC, benötigen wir hier nicht 
%\addtokomafont{chapterentrypagenumber}{\normalfont\textbf}

\addtokomafont{disposition}{\rmfamily}
\addtokomafont{chapterentry}{\textbf}
\RedeclareSectionCommand[tocnumwidth=2.5em]{chapter}
\RedeclareSectionCommand[tocnumwidth=2.5em,tocindent=2.5em]{section}
\RedeclareSectionCommand[tocnumwidth=2.5em,tocindent=5em]{subsection}
%\newcounter{romanchapter} Benötigen wir nur, wenn z.B. Abstract, Literaturverz. und Anhang römisch nummeriert werden soll


%%%%% MATHEMATIK
\usepackage{amssymb, amsmath}
\usepackage{isomath}
\usepackage{pifont} % Für \cmark und \xmark
\newcommand{\cmark}{\ding{51}}%
\newcommand{\xmark}{\ding{55}}%


%%%%% BIBLIOGRAPHIE
\usepackage[style=ieee, mincitenames=1, maxcitenames=1]{biblatex}
\usepackage{url} % Damit URLs in der Quelle schön umgebrochen werden
\setcounter{biburllcpenalty}{7000} % Einstellungen für Packet url
\setcounter{biburlucpenalty}{8000} % Einstellungen für Packet url
\DefineBibliographyStrings{ngerman}{andothers = {{et\,al\adddot}},}
\addbibresource{bib.bib} %
\usepackage{csquotes}
%\emergencystretch=1em
\usepackage[final]{microtype}
%\usepackage[expansion, final]{microtype}
%\usepackage{natbib}
%\setcounter{biburlnumpenalty}{9000}
%\setcounter{biburllcpenalty}{9000}
%\setcounter{biburlucpenalty}{9000}


%%%%% ANHANG
\usepackage{appendix}


%%%%% FARBEN
\usepackage[table,xcdraw]{xcolor}
\definecolor{color20}{RGB}{35,35,35}
\definecolor{color25}{RGB}{69,69,69}
\definecolor{color30}{RGB}{80,80,80}
\definecolor{color80}{RGB}{190,190,190}


%%%%% ÜBERSCHRIFTEN DER EBENEN ÄNDERN
\addtokomafont{chapter}{\color{color30}\normalfont\textbf}
\addtokomafont{section}{\color{color30}\normalfont\textbf}
\addtokomafont{subsection}{\color{color30}\normalfont\textbf}
\addtokomafont{caption}{\small\color{color30}\textit}
\addtokomafont{captionlabel}{\small\color{color30}\textit}


%%%%% LAYOUT
\usepackage[left=2cm,right=2cm,top=3cm,bottom=3cm]{geometry}
\setlength\parindent{0pt} %Kein Einzug nach Ebenenbeginn
\usepackage[onehalfspacing]{setspace} %Zeilenabstand 1,5
\RedeclareSectionCommand[beforeskip=20pt,afterskip=20pt]{chapter} %Wenn neues Chapter startet, geht es auf ne neue Seite. Damit Abstand zur Kopfzeile nicht zu groß, beforeskip = 20
\usepackage[justification=justified,labelfont=bf,format = plain]{caption}


%%%%% KOPF- & FUẞZEILE
\usepackage[headsepline,automark]{scrlayer-scrpage}
\pagestyle{scrheadings}
\clearscrheadfoot
\clearscrplain 
\ihead{\headmark}
\ofoot{\pagemark}
\renewcommand*\chapterpagestyle{scrheadings} %K.&F.zeile auch bei Chapterbeg.


%%%%% BILDER
\usepackage{graphicx}
\usepackage[export]{adjustbox}
\usepackage[section]{placeins}
\let\Oldsection\section
\renewcommand{\section}{\FloatBarrier\Oldsection}
\let\Oldsubsection\subsection
\renewcommand{\subsection}{\FloatBarrier\Oldsubsection}
\let\Oldsubsubsection\subsubsection
\renewcommand{\subsubsection}{\FloatBarrier\Oldsubsubsection}
%\usepackage{here} % mit H in includegrafix wird Bildpos gezwungen


%%%%% TABELLE
\usepackage{multirow}
\usepackage{tabularx} % Nachfolgende 3 Befehle, damit Tabellen Spaltengröße definiert werden kann. z.B. nicht mehr {ccc} sondern {C{2cm}C{2cm}C{2cm}}, Folgende befehle zur neu Definition:
\newcolumntype{L}[1]{>{\raggedright\arraybackslash}p{#1}}
\newcolumntype{C}[1]{>{\centering\arraybackslash}p{#1}}
\newcolumntype{R}[1]{>{\raggedleft\arraybackslash}p{#1}}
\usepackage{longtable} %Mehrseitige Tabellen (Abkürzungsverz.)
\setlength{\tabcolsep}{0.5em} % for the horizontal padding
\renewcommand{\arraystretch}{1.2}% for the vertical padding


%%%%% QUICK COMMANDS
\newcommand{\qm}[1]{\glqq#1\grqq{}} %Anfzeichen
\newcommand{\gradC}[1]{#1$^\circ C$}
\newcommand{\abs}[1]{\lvert #1 \rvert}
\newcommand{\highlight}[1]{\textbf{\textcolor{red}{#1}}}
\newcommand{\finalize}[1]{\textcolor{red}{#1}}
\newcommand{\Abb}[1]{\autoref{fig:#1}}

%%%%% VERLINKUNG
\usepackage[hidelinks,hypertexnames=false]{hyperref}
\hypersetup{pdftitle={CSE Projektarbeit, Pirwass und Müller}}

\begin{document}
\begin{titlepage}
    \begin{center}
		%%%%%DOPPELBILD ANFANG
		\begin{minipage}[b]{\linewidth}
			\centering
			\begin{minipage}[b]{.4\linewidth}
				\includegraphics[width=.8\linewidth, left]{./images/logo_uu.png}
			\end{minipage}
			\hspace{.1\linewidth}% Abstand zwischen Bilder
			\begin{minipage}[b]{.4\linewidth}
				\includegraphics[width=.6\linewidth, right]{./images/logo_thu.png}
			\end{minipage}
		\end{minipage}
		%%%%%DOPPELBILD ENDE
        
		\vspace{3cm}

        \Huge
		\textbf{Windprognose mit neuronalen Netzen}
            
        \vspace{1.5cm}
        \large
        \textbf{Seminararbeit im Kooperationsstudiengang}\\
		\textbf{Computational Science and Engineering Master}
    \end{center}        
	\vfill
	\large	
	\textbf{Erstellt von:}

	Daniel Müller (1085380)\\
	Alicia Pirwass (1085100)

	\vspace{1cm}
	\textbf{Unter der Leitung von:}

	Professor Dr. Stephan Schlüter

	\vspace{1cm}
	\textbf{Abgabedatum:}

	

	\today
            
    
\end{titlepage}
\tableofcontents
%%%%%%%%%%%%%%%%%%%%%%%%%%%%%%%%%%%%%%%%%%%%%%%%%%%%%%%%%%%%%%%%%%%%%%%
%                                                                     %
%                                                                     %
%                                                                     %
%                               KAPITEL                               %
%                                                                     %
%                                                                     %
%                                                                     %
%%%%%%%%%%%%%%%%%%%%%%%%%%%%%%%%%%%%%%%%%%%%%%%%%%%%%%%%%%%%%%%%%%%%%%%
\chapter{Einleitung}

Die Windenergie leistete in den vergangen Jahren einen stetig wachsenden Beitrag zur deutschen Stromerzeugung. 
Seit dem Jahr 2015 \qm{ist die Stromerzeugung aus Wind [...] um mehr als 60 Prozent angestiegen} \highlight{Quelle Umweltbundesamt, hilfe wie kann ich hier Gänsefüßchen machen?}
Laut dem Statistischen Bundesamt wurden im Jahr 2020 24\% des Gesamtstrommixes in Deutschland aus Windenergie gewonnen. \highlight{vgl. Bild von Strommix} 
Das macht den größten Anteil der erneuerbaren Energien aus \highlight{vgl. Umweltbundesamt}
Jedoch ist das Stromnetz zu veraltet um mit diesen Entwicklungen, hauptsächlich den leistungsstarken Windparks auf See, 
mithalten zu können. Die Netze \qm{gelangen an die Grenzen ihrer Übertragungskapazität} \highlight{Quelle windenergie.de}. 
Um diesen Netzengpässen entgegenzuwirken können neben der Erneuerung des Netzes und der Regelung erneuerbarer Erzeugungsanlagen 
des Betreibers Netzoptimierungsmaßnahmen durchgeführt werden. Hierbei wird versucht das bestehende Netz effizient zu nutzen. 
Eine möglicher Ansatz der Netzoptimierung ist das Freileitungsmonitoring.

%%%%%BILD ANFANG
\begin{figure}[tph]
	\begin{center}
		\includegraphics[width=.8\textwidth]{./images/bruttostromerzeugung-erneuerbare-energien.png}
		\caption{Bruttostromerzeugung im Jahr 2020 in Deutschland}
		\label{fig:strommix_deutschland}
	\end{center}
\end{figure}
%%%%%BILD ENDE

Durch eine Überwachung der Windgeschwindigkeit und der Umgebungstemperatur kann daraus die zulässige Übertragungskapazität ermittelt werden. 
Leiterseile halten bei hoher Windgeschwindigkeit und gleichzeitig niedrigen Temperaturen höhere Belastungen aus. 
Es kann also mehr Strom übertragen, wen mehr Windstrom erzeugt wird.\highlight{vgl Quelle windenergie.de}
Aus diesem Grund ist es von großem Interesse Vorhersagen über die Windgeschwindigkeit treffen zu können, um 
die Nutzung des Stromnetzes zu optimieren. In dieser Seminararbeit soll ein Modell entwickelt werden, das 
Windgeschwindigkeit und -richtung eine bis 24 Stunden im Voraus vorhersagen kann. Es sollen in dieses Modell neben zeitlich diskretisierten 
meteorologischen Daten auch eine örtliche Diskretisierung einfließen, welche in dieser Arbeit validiert werden soll.

\section{Einordnung in die Literatur}

\highlight{MFSTC in Ausblick, Erklären warum Daten der Standorte einfach übergeben (wenige Standorte?), Warum nur stündliche Daten? (Ausblick kleinere Intervalle der Daten?), Ausblick statt RNN CNN??\bigskip}

In der bestehenden Literatur im Bereich der Windgeschwindigkeits-, beziehungsweise Wettervorhersage, ist eine Entfernung von 
deterministischen Modellen \cite{1963_Lorenz_DeterministicNonperiodicFlow} hin zu Zeitreihenanalyse auf Basis historischer Messdaten erkennbar. 
Innerhalb der letzten zwei Dekaden sind einige Ansätze in der Zeitreihenanalyse zu differenzieren. Klassische Methoden in der 
Zeitreihenprognose, wie das ARMA- und ARIMA-Modell \cite{2016_Cadenas_WindSpeedPrediction} oder auch 
die Weiterentwicklung dessen, mit Berücksichtigung eines saisonalen Einflusses, das SARIMA-Modell 
\highlight{Es fehlt noch Meng 2010; Im Ordner Literatur}\cite{2018_Alencar_HybridApproachCombining,2019_TenaGarcia_ForecastDailyOutput,2019_Haddad_WindSolarForecasting,2002_Igboekwe_StochasticSimulationHourly,2012_MuhammadSami_PredictionRateDust} sind 
nicht nur ein häufig verwendetes Vorhersagemodell, sondern werden gerade deshalb auch häufig als Vergleichsmodell herangezogen 
\highlight{Es fehlt Kreuzer et al. 2020; Im Ordner Literatur}\cite{2012_Cao_ForecastingWindSpeed,2019_Chen_MultifactorSpatiotemporalCorrelation}. Darüber hinaus existieren das VAR-, VARTA und auch VARTAX-Modelle 
\highlight{Es fehlt Orpia et al. 2014, Dowell 2015}\cite{2007_Ewing_TimeSeriesAnalysis,2015_He_SparsifiedVectorAutoregressive,2016_Koivisto_WindSpeedModeling}, 
welche im Gegensatz zu ARMA-Modellen mehrere Variablen besitzen. Auch wird es versucht das ARIMA-Modell mit Clustermethoden 
zu verbessern oder mit klassischen Machine Learning Konzepten zu kombinieren \cite{2017_Zhang_HybridMethodShortTerm,2011_Guo_CorrectedHybridApproach}. 
Zunehmend werden auch Neuronale Netze, welche in letzter Zeit stark an Popularität zugenommen haben, für Problemstellungen dieser 
Art verwendet. Neben Ansätzen basierend auf einfachen vorwärtsgerichteten Neuronalen Netzen \cite{2019_Samet_EvaluationNeuralNetworkbased,2017_Chang_ImprovedNeuralNetworkbased}, Convolutional Neural Networks (CNN) \cite{2020_Zhao_ShorttermAverageWind,2019_Chen_MultifactorSpatiotemporalCorrelation},  
Generative Modelle \cite{2019_Khodayar_IntervalDeepGenerative} und Evolutionäre Modelle \cite{2012_Wang_ShorttermWindSpeed} 
waren Ansätze, die auf Recurrent Neural Networks (RNN) basieren auffällig oft vertreten. Hierbei waren RNNs mit Long- Shortterm Memory Units (LSTM) \cite{2018_Dong_WindPowerPrediction,2020_Delgado_WindTurbineData,2020_Moharm_WindSpeedForecast,2019_Prabha_WindSpeedForecasting,2019_Cali_ShorttermWindPower} oder auch LSTM Zellen in Kombination mit 
anderen Technologien \cite{2019_Chen_MultifactorSpatiotemporalCorrelation,2016_Allende_RecurrentNetworksWind,2018_Liu_WindSpeedForecasting,2018_Yao_MultidimensionalLSTMNetworks} sehr prominent.

Drei dieser Ansätze werden hinsichtlich der Datenbasis, des Modellaufbaus und der Modellgüte näher betrachtet. \citeauthor{2020_Delgado_WindTurbineData} \cite{2020_Delgado_WindTurbineData} und 
\citeauthor{2018_Yao_MultidimensionalLSTMNetworks} \cite{2018_Yao_MultidimensionalLSTMNetworks} verwendeten Daten eines Standorts in je der Nordtürkei und Westchina in Intervallen von 
zehn Minuten von je einem Jahr und drei Jahren. \citeauthor{2020_Delgado_WindTurbineData} versuchte mithilfe der generierten Leistung einer Windturbine, der theoretischen Leistung,  
der gemessenen Windgeschwindigkeit und Windrichtung diese vier Parameter der darauf folgenden 10 Minuten zu prognostizieren. Dagegen floss in 
das von \citeauthor{2018_Yao_MultidimensionalLSTMNetworks} entwickelte Modell statt der Leistung der Windturbine die Temperatur ein. \citeauthor{2019_Chen_MultifactorSpatiotemporalCorrelation} 
\cite{2019_Chen_MultifactorSpatiotemporalCorrelation} verwendete Messdaten eines halben Jahres in fünfminütigen Intervallen aus der Region Texas in den USA. 
Es wurden hierbei jedoch nicht nur deutlich mehr unterschiedliche Kenngrößen (Windgeschwindigkeit, Windrichtung, Temperatur, Taupunkt, Temperatur, Windböen, Messhöhe und relative Luftfeuchtigkeit) 
als in den anderen beiden Ansätzen verwendet, sondern auch eine Mehrzahl an Standorten.

Bei dem Modell nach \citeauthor{2020_Delgado_WindTurbineData} handelt es sich um ein RNN mit 65 LSTM Zellen. Die Lossfunktion ist der Mean Absolute Error (MAE), der 
verwendete Optimizer lautet Adam und die Batch Size beträgt 15. Insgesamt wurden 21 Epochen durchlaufen. Es wurde das Single Step Verfahren verwendet. Für die Ergebnisse, monatliche Mittelwerte, dieses Modells wurden der MAE, 
Mean Square Error (MSE) und das Bestimmtheitsmaß ($R^2$) ermittelt. Der MAE der Windgeschwindigkeit liegt in einem Bereich von 0.015m/s bis 0.027m/s. Diese sehr geringen Werte stammen 
mutmaßlich daher, dass in diesem Modell sehr kleine Intervalle der Eingangsdaten mit einer Single Step Prognose kombiniert wurden. 

Im Modell von \citeauthor{2018_Yao_MultidimensionalLSTMNetworks} wurde bei den Eingangsdaten zunächst eine Fuzzy-rough set Faktorreduktion (FRS) durchgeführt. Das RNN, in das die vorverarbeiteten Daten 
einfließen ist folgendermaßen aufgebaut. Das Netz hat drei Layer mit 108 LSTM Zellen und arbeitet mit der Aktivierungsfunktion RELU. Das Modell führt im Gegensatz zum vorherigen beschrienen Modell eine 
rolling prediction durch. Der Input sind jeweils die Daten von zehn Tagen, womit eine Vorhersage von 24 Stunden in die Zukunft getroffen werden soll. 
Eine Cut-off Rate von 0.2 wurde verwendet und die Learning Rate beträgt 0.001. Verwendet wurde der Optimizer RMSprop und das Fehlermaß waren der MAE und der Mean Average Percentage Error (MAPE). 
Um das Modell zu vergleichen wurden neben dem entwickelten Modell dasselbe ohne vorherige Faktorreduktion der Daten und ein Neuronales Netz, welches als Lernverfahren Back Propagation verwendet, entwickelt. 
Aus den Ergebnissen dieser drei Modelle wurden der MAE, MAPE und der maximale Fehler berechnet. Das entwickelte Modell schnitt in diesem Vergleich am besten ab. Im Vergleich zum vorigen Modell ist der MAE höher, 
wobei eine mittlere Abweichung von ungefähr 0.5m/s vertretbar ist.

Das Modell von \citeauthor{2019_Chen_MultifactorSpatiotemporalCorrelation} ist für diese Seminararbeit besonders interessant. Der Einfluss mehrerer Standorte auf die Prognose der Windgeschwindigkeit eines Standorts ist neuartig. 
Um die örtliche mit der zeitlichen Diskretisierung in den Eingabedaten zu verknüpfen und darüber hinaus Informationen über die Korrelation zwischen Zeit, Ort und Naturphänomen herzustellen, wurden die Messdaten mithilfe 
des Mathematical representation of multi factor spatio-temporal correlation model (MFSTC) in einer 3D-Matrix dargestellt. Diese Werte werden von einem CNN mit LSTM Zellen verarbeitet. Das Netz hat eine Batch Size von 128, einen 
Regularisierungskoeffizient von 0.34 und eine Learning Rate von 0.3. 2000 Epochen wurden insgesamt durchlaufen. Verwendet wurde der Optimizer Adam und die Lossfunktion war der MSE. Der Vergleich der Ergebnisse und die Einordnung dieser 
im Vergleich zu anderen Modellen war in dieser Arbeit sehr ausführlich. In drei Experimenten mit jeweils unterschiedlichen Testdaten wurden acht Fehlermaße von variiernden Modellen verglichen. Die Fehlermaße sind die Residuenquadratsumme (SSE), 
der MAE, der RMSE, der Sub-Divisional Error (SDE), der Theilsche Projektionskoeffizient (U1), der Index of agreement (IA), die Direction accuracy (DA) und zuletzt der Pearson correlation coefficient (PCC). Das erste Experiment zielte darauf ab 
die Sinnhaftigkeit der Verwendung des MFSTC Modells und daraufhin die Kombination eines CNNs mit LSTM Zellen mit vorheriger Anwendung des MFSTC Modells zu validieren. Hierbei wurden zunächst ein CNN ohne LSTM Zellen, dasselbe mit Anwendung des 
MFSTC Modells, ein CNN mit LSTM Zellen und das entwickelte Modell verglichen um MFSTC zu validieren und im Anschluss ein CNN ohne LSTM Zellen, mit LSTM Zellen, ein MFSTC-CNN ohne LSTM Zellen und das entwickelte Modell verglichen. 
Im zweiten Teil des ersten Experiments wurden ein ARIMA Modell und ein Multi Layer Perceptron (MLP) zusätzlich als Benchmark verwendet. Das entwickelte Modell schnitt im Vergleich zu den anderen Modellen am besten ab, womit die Sinnhaftigkeit der Verwendung des 
MFSTC Modells und auch des entwickelten Modells validiert werden konnten. Der MAE des Modells war in einem Bereich von 1m/s. 
Im zweiten Experiment wurden nun einige unterschiedliche Neuronale Netze miteinander verglichen. Neben einer Linearen Regression (LR), einem MLP, einem CNN, einem LSTM und einer Kombination aus dem MFSTC Modell mit einer CNN-MLP Fusion und MLP-LSTM Fusion, 
schnitt das von \citeauthor{2019_Chen_MultifactorSpatiotemporalCorrelation} entwickelte Modell wieder am besten ab. Der MAE war nun im Bereich von 2m/s. 
Das dritte Experiment verglich nun das eigene Modell mit üblichen Modellen der Windprognose und den Modellen von drei anderen Veröffentlichungen. Wieder schnitt das entwickelte Modell am besten ab. 
Zuletzt wurde mit all den erhaltenen Ergebnissen der Diebold-Mariano Test durchgeführt, eine übliche Hypothesen Testmethode, bei der die Tauglichkeit des vorgestellten Modells gegenüber den anderen Methoden endgültig validiert wird.

Anhand dieser drei vorgestellten Paper ist erkenntlich, dass Windprognosemodelle, die dem Stand der Technik entsprechen, nicht nur eine stark voneinander abweichende Datenbasis besitzen, sondern auch die Modelle, die sich alle der Technologie von LSTM Zellen 
zu Nutze machen, trotzdem stark voneinander abweichen. Aufgrund dieser großen Vielfalt an Freiheitsgraden ist die Validierung und Einordnung von Windprognosemodellen sehr komplex und in der vorhandenen Literatur keinesfalls einheitlich.

\highlight{
	Soll hier noch das eigene geplante Modell erwähnt werden? Möglicherweise auch, dass das Modell von Chen et al. den neuen Ansatz von mehrere Standorten verfolgt, was ja auch das Ziel in dieser Arbeit ist.
}



%%%%%%%%%%%%%%%%%%%%%%%%%%%%%%%%%%%%%%%%%%%%%%%%%%%%%%%%%%%%%%%%%%%%%%%
%                                                                     %
%                                                                     %
%                                                                     %
%                               KAPITEL                               %
%                                                                     %
%                                                                     %
%                                                                     %
%%%%%%%%%%%%%%%%%%%%%%%%%%%%%%%%%%%%%%%%%%%%%%%%%%%%%%%%%%%%%%%%%%%%%%%
\chapter{Grundlagen}

\section{Neuronale Netze zur Zeitreihenprognose}
\highlight{ein paar Grundlagen zu neuronalen Netzen; 
Beschreibung versch Architekturen}

\section{Das Benchmarkmodell}
\highlight{Das SARIMA-Modell erklären, falls wir es zukünftig noch als Benchmark verwenden}

\section{Zirkuläre Daten}\label{section:circ_data}

\highlight{
	Beispiele was alles zyklisch sein kann
	Darstellung Winkel in Grad, oder Tageszeit in Stunden lässt den zyklischen Aspekt der Daten verloren gehen 
	Gerade im Kontext von Machine learning ist es sinnvoll, wenn das Modell den zyklischen Zusammenhang der Daten versteht
	Versuch mithilfe einer Transformation zyklischen Aspekt darstellen zu können
	Üblich bei Zeitangaben diese in eine sinus und cosinus fkt zu zerlegen.
	}



Winkel werden häufig auf einer Skala von $0^\circ$ bis $360^\circ$ angegeben. Für den Menschen mag das sehr intuitiv sein, ein neuronales Netz muss jedoch erlernen, dass die Intervallgrenzen den selben Winkel beschreiben.

Die Zeitmessung ist ebenso zirkulär. Sowohl Sekunden, Minuten und Stunden als auch Wochentage und Monate werden beim Überschreiten der oberen Intervallgrenze auf die untere zurückgesetzt. 

%%%%%%%%%%%%%%%%%%%%%%%%%%%%%%%%%%%%%%%%%%%%%%%%%%%%%%%%%%%%%%%%%%%%%%%
%                                                                     %
%                                                                     %
%                                                                     %
%                               KAPITEL                               %
%                                                                     %
%                                                                     %
%                                                                     %
%%%%%%%%%%%%%%%%%%%%%%%%%%%%%%%%%%%%%%%%%%%%%%%%%%%%%%%%%%%%%%%%%%%%%%%

\chapter{Daten}

\section{Datenbasis}\label{section:datenbasis}
Zur Erstellung des Modells wurde eine Kombination aus gemessenen Daten von Stationen des Deutschen Wetterdienstes (DWD) sowie künstlich generierten Daten verwendet.
Während die gemessenen Daten die aktuelle Wetterlage an einigen Standorten Deutschlands beinhalten, geben die generierten Daten darüber hinaus weitere Informationen zu saisonalen und täglichen Abhängigkeiten. Die Zusammensetzung des Rohdatzensatzes kann der Abbildung \ref{fig:rohdatensatz} entnommen werden. Nach der in Kapitel \ref{section:Datenaufbereitung} beschriebenen Datenaufbereitung kann dieser Datensatz zum Training des neuronalen Netzes genutzt werden.

%%%%%BILD ANFANG
\begin{figure}[tph]
	\begin{center}
		\includegraphics[]{./images/rohdatensatz.pdf}
		\caption{Zusammensetzung des Rohdatensatzes}
		\label{fig:rohdatensatz}
	\end{center}
\end{figure}
%%%%%BILD ENDE

\subsection{Reale Messdaten}
Die realen Messdaten werden vom Deutschen Wetterdienstes über einen Webserver kostenlos zur Verfügung gestellt. In dieser Arbeit sind die Daten in stündlicher Auflösung von vier Messtationen (Münsingen-Apfelstetten, Stötten, Günzburg und Laupheim) im Ulmer Umkreis zum Einsatz gekommen. Die genaue geographische Lage kann der Abbildung \ref{fig:map} entnommen werden. Der Deutsche Wetterdienst betreibt im Ulmer Umkreis zwar noch weitere Wetterstationen, jedoch messen diese die Windgeschwindigkeit und Windrichtung nicht. Insgesamt umfasst der Datensatz die Jahre \highlight{2008?} bis 2021. Davon sind jedoch \highlight{X\%} Datenlücken. Diese Lücken werden in Kapitel \ref{section:Datenlücken} genauer beleuchtet. 

%%%%%BILD ANFANG
\begin{figure}[tph]
	\begin{center}
		\includegraphics[width=\textwidth]{./images/map.png}
		\caption{Geographische Lage der Messstationen \cite{2021_OpenStreetMap-Contributors_UMap}}
		\label{fig:map}
	\end{center}
\end{figure}
%%%%%BILD ENDE

Zur Prognose der Windgeschwindigkeit und Windrichtung können neben deren selbst \highlight{(Satzbau?)} auch andere Messgrößen wie Sonnenscheindauer, Luftdruck, Temperatur und Luftfeuchtigkeit genutzt werden. Jedoch wird nicht jede dieser Größen an jeder Messstation bestimmt. Die folgende Tabelle \ref{tab:messgrößen} gibt einen detaillierteren Aufschluss über die tatsächlich gemessenen Größen.

%%%%%TABELLE ANFANG
\begin{table}[ht]
	\centering
	\caption{Gemessene Größen je Messstation \highlight{Auflösung der einzelnen Größen ergänzen. Wird referenziert!}}
	\begin{tabular}{|l|c|c|c|c|}
		\hline
        \rowcolor{color80}
		& \textbf{Münsingen-Apfelstetten} & \textbf{Stötten} & \textbf{Günzburg} & \textbf{Laupheim} \\\hline
		Windgeschwindigkeit & \cmark & \cmark & \cmark & \cmark \\\hline
		Windrichtung & \cmark & \cmark & \cmark & \cmark \\\hline
		Sonnenscheindauer & \cmark & \cmark & \xmark & \xmark \\\hline
		Luftdruck & \xmark & \cmark & \xmark & \cmark \\\hline
		Temperatur & \cmark & \cmark & \cmark & \cmark \\\hline
		Luftfeuchtigkeit & \cmark & \cmark & \cmark & \cmark \\\hline
	\end{tabular}
\label{tab:messgrößen}
\end{table}
%%%%%TABELLE ENDE

Um einen tieferen Einblick in das Verhalten des Windes an der jeweiligen Messstation zu erhalten, bieten sich Windrosen zur Visualisierung an. Dabei wird die Häufigkeit der Windgeschwindigkeit in Abhängigkeit der Windrichtung radial dargestellt. Der Abbildung \ref{fig:windroses} können die Windrosen für die vier ausgewählten Messstationen entnommen werden. Wie in den südlichen Regionen Deutschlands zu erwarten \highlight{satzbau?}, ist die vorherschende Windrichtung West bis Südwest. 

Bei detaillierter Analyse fällt jedoch auf, dass der Wind an der Messstationen Münsingen-Apfelstetten besonders bei Leichtwind sehr volatil ist. Der Grund hierfür könnte in der Tatsache begründet sein, dass sich die Messstation in einem flachen Tal mit Nord-Südlicher Ausrichtung befindet. Desweiteren lässt sich erkennen, dass unter den vier Messstationen in Stötten die stärksten Winde auftreten. 

%%%%%BILD ANFANG
\begin{figure}[tph]
	\begin{center}
		\includegraphics[]{./images/windroses.pdf}
		\caption{Windrosen der vier ausgewählten Messstationen}
		\label{fig:windroses}
	\end{center}
\end{figure}
%%%%%BILD ENDE

Im Folgenden wird versucht, die Windgeschwindigkeit \highlight{und Windrichtung?} für die Messstation in Günzburg \highlight{und Stötten?} zu prognostizieren. 

\highlight{Warum Günzburg? 1. Liegt östlich der anderen Stationen 2. Relativ klare Windrose}

\highlight{Warum Stötten? 1. Alle Messgrößen werden vorort gemessen 2. Stärkster Wind}

\subsection{Generierte Messdaten}\label{section:gen_daten}
Die Wetterlage in Deutschland weist wiederkehrende Trends innerhalb eines Tages und auch innerhalb eines Jahres auf. Offensichtlich ist es am Tag wärmer als in der Nacht und im Sommer wärmer als im Winter. Um diese Information für das Modell nutzbar zu machen, werden zusätzlich zu den gemessenen Daten zwei weitere Variablen eingeführt: \highlight{Warum Sinus? Weil zirkuläre Daten! 24h=0h! Oben gibts das Kapitel \ref{section:circ_data} in dem zirkuläre Daten erklärt werden, d.h. erwähnen!}

\begin{itemize}
	\item \textit{Kalendarischer Tageswert}\\
	Über die folgende Funktion wird die Stunde des Tages $HOD \in [0, \dots, 23]$ in einen Sinuswelle umgewandelt. Das Maximum wird um 12 Uhr, das Minimum um 0 Uhr erreicht. Abgebildet wird auf das Intervall $[0,1]$.
	\begin{equation}\label{eq:cal_day}
		cal_{day} = \frac{1}{2}\Big(\cos \big(\pi (\frac{HOD}{12}+ 1)\big) + 1 \Big)
	\end{equation}
	\item \textit{Kalendarischer Saisonalwert}\\
	Die folgende Funktion transformiert den Tag des Jahres $DOY \in [1, \dots, 365]$ in eine Sinuswelle. Schaltjahre werden nicht berücksichtigt. Außerdem wird die Stunde des Tages $HOD \in [0, \dots, 23]$ mit einbezogen. Diese spielt für den Funktionswert jedoch nur eine untergeordnete Rolle. Das Maximum wird am 1.Juli, das Minimum am 1.Januar erreicht. Die Funktion bildet ebenfalls auf das Intervall $[0,1]$ ab.
	\begin{equation}\label{eq:cal_seas}
		cal_{seas} = \frac{1}{2}\Big(\cos\big(\pi (\frac{DOY -1 + \frac{HOD}{24}}{\frac{365}{2}}+ 1)\big) + 1\Big)
	\end{equation}
	
\end{itemize}

\highlight{Eventuell erwähnen, das diese Formeln nicht optimal sind. Grund: Höchste Tagestemperatur i.A. nicht um 12 Uhr sondern um 13-14 Uhr, Höchste Jahrestemperatur i.A. nicht am 1.Juli sondern Mitte August. Außerdem ist z.B. $dal_{day}(HOD=6Uhr) = dal_{day}(HOD=18Uhr)$}

\subsection{Analyse der Datenbasis}
Der Korrelationskoeffizient gibt Aufschluss darüber, ob zwischen zwei größen ein linearer Zusammenhang besteht. Er kann als Entscheidungshilfe zur Auswahl der Input-Größen für das neuronale Netz dienen. In Abbildung \ref{fig:corr} wird die Korrelation zwischen den Größen relative Luftfeuchte ($RH$), Sonnenscheindauer ($SD$), Temperatur ($T$), Windgeschwindigkeit ($v$), Windrichtung ($\varphi$) sowie die in Kapitel \ref{section:gen_daten} beschriebenen kalendarischen Daten $cal_{day}$ und $cal_{seas}$ dargestellt. 



Da die Größen Windstärke und \highlight{Windrichtung?} prognostiziert werden sollen, sind die Korrelationen zwischen dieser und anderer Größen natürlich von besonderem Interesse. Aus Abbildung \ref{fig:corr} lässt sich erkennen, dass der Luftdruck in einem Zusammenhang mit der Windgeschwindigkeit steht (Fällt $RH$ steigt $v$). Ebenso lässt sich aus der Korrelation zwischen Windgeschwindigkeit und der Temperatur bzw. dem kalendarischen Saisonalwert $cal_{seas}$ eine erhöhte Windgeschwindigkeit im Winter ablesen. Der kalendarische Tageswert $cal_{day}$ scheint jedoch nicht in einem linearen Zusammenhang mit der Windgeschwindigkeit zu stehen.

\bigskip
An dieser Stelle ist es wichtig zu erwähnen, dass der Korrelationskoeffizient nur einen linearen Zusammenhang zwischen zwei größen abbildet. Komplexere Zusammenhänge werden nicht erkannt \cite{2020_Ranjan_EstimatingNonlinearCorrelation}. Es ist also möglich, dass eine Größe für die Prognose wertvolle Information enthält, jedoch nicht mit der Windgeschwindigkeit korreliert.

%%%%%BILD ANFANG
\begin{figure}[tph]
	\begin{center}
		\includegraphics[]{./images/corr.pdf}
		\caption{Korrelationskoeffizienten für die Messdaten der Station in Stötten \highlight{(Ändern zu Günzburg, da diese prognostiziert wird)} sowie kalendarische Daten im Zeitraum 01.01.2016 bis 31.12.2020.}
		\label{fig:corr}
	\end{center}
\end{figure}
%%%%%BILD ENDE

\highlight{Grafikidee: Windgeschwindigkeit über Uhrzeit}

\section{Datenaufbereitung}\label{section:Datenaufbereitung}
Die Datenaufbereitung ist häufig eines der aufwendigeren Teile eines Machine-Learning Projektes. Daher wird in dieser Arbeit auf diesen Teil vertieft eingegangen. Abbildung \ref{fig:preprocessing} skizziert den Ablauf der Datenaufbereitung. Dabei werden die folgenden fünf Schritte durchlaufen, die in den nachfolgenden Kapiteln detaillierter beschrieben werden:

%%%%%BILD ANFANG
\begin{figure}[tph]
	\begin{center}
		\includegraphics[]{./images/preprocessing.pdf}
		\caption{Ablauf der Datenaufbereitung. \highlight{Beschreiben, dass die grünroten Box der Datensatz ist}}
		\label{fig:preprocessing}
	\end{center}
\end{figure}
%%%%%BILD ENDE

\begin{enumerate}
	\item Im Rohdatensatz gibt es einige kürzere und längere Datenlücken (In Abb. \ref{fig:preprocessing} dargestellt als rote Segmente). Über kurze Datenlücken wird linear interpoliert. Außerdem werden die Winddaten von der polaren in eine kartesische Darstellung umgewandelt.
	\item Aus dem durch Schritt 1. generierten Datensatz werden die Spalten gewählt, die es zu prognostizieren gilt (In Abb. \ref{fig:preprocessing} dargestellt durch verdunkelte Spalten). In dieser Arbeit wird die Windgeschwindigkeit \highlight{(und Windgeschwindigkeit?)} einer Station gewählt.
	\item Der Datensatz wird nun in einen Inputdatensatz und einen Labeldatensatz (In Abb. \ref{fig:preprocessing} verdunkelt) aufgeteilt.
	\item Die verbleibenden längeren Datenlücken im Input- und Labeldatensatz werden \qm{herausgeschnitten}. Übrig bleiben Fragmentpaare, die keine Fehlstellen enthalten.
	\item Der Inputdatensatz (hellgrün) wird normiert. Dieser Schritt erleichtert das Training des neuronalen Netzes.
\end{enumerate}

\subsection{Interpolation über Datenlücken}\label{section:Datenlücken}

In realen Messsystemen kann es immer wieder zu Ausfällen kommen (z.B. vereistes Anemometer oder verschmutzte Sensoren). Auch kann es bei der Übertragung oder Speicherung der Daten zu Fehlern kommen. Der Deutsche Wetterdienst hat daher ein mehrstüfiges Validationssystem und kennzeichnet Fehler entsprechend. In den Rohdaten (siehe Kap. \ref{section:datenbasis}) ist dies meist als Wert $-999$ gekennzeichnet. Es kann sich also darauf verlassen werden, dass die verbleibenden \qm{wahren} Daten glaubhaft sind. 

Datenlücken müssen vor dem Training mit dem neuronalen Netz entfernt werden. In Abhängigkeit der Länge des Zeitraums $\Delta t$, über den sich eine Datenlücke erstreckt, wird die Datenlücke entfernt:

\begin{itemize}
	\item $\Delta t \leq 4h$
	
	Tritt eine kurze Datenlücke auf, so werden die fehlenden Werte durch eine lineare Interpolation anhand der direkt umschließenden wahren Werte ergänzt.

	\item $\Delta t > 4h$

	Eine Interpolation längerer Daten ist nicht mehr sinnhaft. Daher wird der Datensatz, wie in Abbildung \ref{fig:preprocessing} im vierten Schritt verdeutlicht, entlang großer Datenlücken aufgeteilt.

\end{itemize}

\highlight{Grafik wo und wie lange es im Datensatz Lücken gibt --> Heatmap}

\subsection{Winddaten als Vektoren}
Die Windmessung erfolgt (von modernen Laser- oder Schallmessungen abgesehen) üblicher Weise mithilfe einens Schalenanemometers und einer Windfahne. Zu jedem Messzeitpunkt werden also zwei Werte aufgenommen. Die Windgeschwindigkeit $v$ wird in $m/s$ gemessen. Die Windrichtung $\varphi$ wird typischer Weise in Grad angegeben, es gilt also $\varphi \in [0,360)$. Norden wird dabei mit $\varphi_{north} = 0^\circ$ gewählt. Dieses $(v,\varphi)$-Tupel lässt sich als Polarkoordinate interpretieren. Jedoch tritt hier das in Kapitel \ref{section:circ_data} beschriebene Problem der zirkulären Daten auf, da $\varphi_{north} = 0^\circ = 360^\circ$ gilt.

Es gilt zu belegen, dass sich dieses Problem umgehen lässt, indem der Wind nicht in polaren sondern kartesischen Koordinaten dargestellt wird. Der Windvektor, der durch das $(v,\varphi)$-Tupel beschrieben ist, wird so in eine $v_x$ und eine $v_y$ Komponente zerlegt. Dieses Vorgehen ist in Abbildung \ref{fig:pol2cart} skizziert.

\highlight{Gleichung pol2cart?}

%%%%%BILD ANFANG
\begin{figure}[tph]
	\begin{center}
		\includegraphics[scale = 1]{./images/pol2cart.pdf}
		\caption{Interpretation der Windgeschwindigkeit und Windrichtung als Vektorkomponenten}
		\label{fig:pol2cart}
	\end{center}
\end{figure}
%%%%%BILD ENDE

\highlight{Evtl erwähnen, dass wenn diese Umwandlung genutzt wird, nach dem NN wieder zurück umgewandelt wird. d.h. $(v,\varphi) \rightarrow (v_x,v_y) \rightarrow(v,\varphi)$}

Die hier beschriebene Umwandlung kann der Abbildung \ref{fig:pol2cart2} anhand realer Messdaten entnommen werden. Im oberen Teil ist die Häufigkeitsverteilung in $(v,\varphi)$-Form dargestellt. Die zirkuläre Eigenschaft $\varphi_{north} = 0^\circ = 360^\circ$ ist hier nicht sichtbar. Die Häufigkeitsverteilung für die umgewandelten Koordinaten in $(v_x,v_y)$-Form hingegen ist klar um den Ursprung verteilt. Hier lässt sich die zirkuläre Eigenschaft gut erkennen. Das strahlenartige Muster erklärt sich durch die diskrete Angabe der Windrichtung in $10^\circ$ Schritten. Bei einer feineren Auflösung wäre die Verteilung weniger verzerrt.

%%%%%BILD ANFANG
\begin{figure}[tph]
	\begin{center}
		\includegraphics[scale = 1]{./images/pol2cart_visualize.pdf}
		\caption{Verteilung des Windes bei $(v,\varphi)$ bzw. $(v_x,v_y)$-Interpretation für die Messtation Ulm-Mähringen \highlight{(Ändern zu Günzburg, da diese prognostiziert wird)} im Zeitraum 01.2016 bis 03.2021 \highlight{FEHLER IN GRAFIK: x-Label Oben wird überdeckt}}
		\label{fig:pol2cart2}
	\end{center}
\end{figure}
%%%%%BILD ENDE

\subsection{Aufteilung in Input- und Labeldatensatz}
\highlight{
	\begin{itemize}
		\item Warum teilen?
		\item Größen konkret nennen
		\item Mit Grafik erklären
	\end{itemize}
}

\subsection{Normierung}

\highlight{Erklärung, warum Normierung nötig ist. Tabelle überdenken.}

%%%%%TABELLE ANFANG
\begin{table}[ht]
	\centering
	\caption{Normalisierung der Eingabedaten}
	\begin{tabular}{|c|c|c|c|c|c|c|c|}
		\hline
		\rowcolor{color80}
		\textbf{$X$} & \textbf{$w_x$} & \textbf{$w_y$} & \textbf{$p$} & \textbf{$T$} & \textbf{$RH$} & \textbf{$cal_{day}$} & \textbf{$cal_{seas}$} \\ \hline
		Einheit $X$ & $m/s$ & $m/s$ & $hPa$ & $^\circ C$ & \% & $1$ & $1$ \\ \hline
		$min(X)$       & $-22.1$ & $-13.7$ & $975.2$ & $-19.1$ & $10$ & $0$ & $0$ \\ \hline
		$max(X)$       & $13.5$ & $9.7$ & $1047.3$ & $34.4$ & $100$ & $1$ & $1$ \\ \hline
		$min(norm(X))$ & $-1$ & $-1$ & $0$ & $0$ & $0$ & $-$ & $-$ \\ \hline
		$max(norm(X))$ & $1$ & $1$ & $1$ & $1$ & $1$ & $-$ & $-$ \\ \hline
	\end{tabular}
\label{tab:normalisierung}
\end{table}
%%%%%TABELLE ENDE

\section{Generierung der Trainingsbeispiele}
Nachdem die Rohdaten aufgearbeitet worden sind, gilt es nun tatsächliche Trainingsbeispiele (Samples) zu generieren. Dieser Prozess wird in Abbildung \ref{fig:sampling} dargestellt, wobei die Grafik als Fortsetzung von Abbildung \ref{fig:preprocessing} verstanden werden kann. Das Ziel des Samplings ist die Generierung eins Pools aus Trainingsbeispielen, womit das neuronale Netz angelernt werden kann.

Zunächst müssen die Größen der Moving-Windows (siehe Abb. \ref{fig:sampling})festgelegt werden. Bei den Inputdaten gibt die Größe $n$ des Moving-Windows an, wieviele historische Zeitschritte für die Prognose genutzt werden. Das heißt, bei einer Prognose zum Zeitpunkt $t_0$ werden die Daten zu den Zeitpunkten $t_{n-1}, \dots, t_0$ als Inputdaten verwendet. Die Größe $m$ des Moving-Window auf den Labeldaten gibt den Prognosehorizont an. Das Ziel ist es also bei einer Prognose zum Zeitpunkt $t_0$ die Windgeschwindigkeiten \highlight{(und Windrichtungen?)} zu den Zeitpunkten $t_1, \dots, t_m$ vorherzusagen.

Das Ergebnis der Datenaufbereitung, beschrieben im Kapitel \ref{section:Datenaufbereitung}, sind datenlückenfreie Fragmentpaare. Ein solches Fragmentpaar besteht aus einem Input- und einem Labeldatensatz. Darüber werden nun die im vorherigen Schritt definierten Moving-Windows \qm{geschoben}. Die Schrittweite wird dabei kleinstmöglich gewählt, so entstehen am meisten Samples. Das Sampling wird auf jedem Fragmentpaar angewandt.

%%%%%BILD ANFANG
\begin{figure}[tph]
	\begin{center}
		\includegraphics[]{./images/sampling.pdf}
		\caption{\highlight{{TODO}}}
		\label{fig:sampling}
	\end{center}
\end{figure}
%%%%%BILD ENDE

\highlight{
	An dieser Stelle feht noch, dass die Samples in drei Teile getrennt werden:
	\begin{itemize}
		\item Warum trennen?
		\item Aufteilung in Trainings (65\%), Validierung(30\%) und Testdaten (5\%)
		\item Wieviele Samples haben wir?
	\end{itemize}
}

\chapter{Implementation}
\section{Benchmarkmodell}
\highlight{
	\begin{itemize}
		\item Welche gibt es?
		\item Warum Benchmark?
		\item Warum sind so Modelle schwierig zu vergleichen? Überall unterschiedliche Daten!
	\end{itemize}
}

\section{Festlegung der Modellarchitektur \highlight{(Arbeitstitel)}}
\highlight{Begründen aus welchen Gründen wir uns für diese drei Methoden entschieden haben 
(RNNseq2seq, RNNseq2vec, DNN) z.B. Datengrundlage, Zeitaspekt, bla; 
Tabelele: Welche Lossfunktion, Metric, Prognosehorizont, Historie, tatsächliche Architektur (Schichten, Neuronen), 
welche Regularisierung}

\section{Trainingsphase und Optimierung}
\highlight{Random search über Konfigurationen aus Tabelle und daraus optimales NN; optional Grafik über Fehler, Loss, und so}

\section{Ergebnisse}

\highlight{Hier nur Performance der verschiedenen Netze. Analyse der Ergebnisse kommt im nächsten Kapitel}

%%%%%%%%%%%%%%%%%%%%%%%%%%%%%%%%%%%%%%%%%%%%%%%%%%%%%%%%%%%%%%%%%%%%%%%
%                                                                     %
%                                                                     %
%                                                                     %
%                               KAPITEL                               %
%                                                                     %
%                                                                     %
%                                                                     %
%%%%%%%%%%%%%%%%%%%%%%%%%%%%%%%%%%%%%%%%%%%%%%%%%%%%%%%%%%%%%%%%%%%%%%%

\chapter{Diskussion und Ausblick}

\section{Diskussion der Ergebnisse}
\highlight{Ergebnisse in Kontext setzen. War unser Modell vergleichsweise gut?}

\subsection{Detaillierte Prognosegüte}

%%%%%BILD ANFANG
\begin{figure}[tph]
	\begin{center}
		\includegraphics[width=0.9\linewidth]{./images/güte_über_zeit.png}
		\caption{\highlight{Grafik nicht final!}}
		\label{fig:güte_über_zeit}
	\end{center}
\end{figure}
%%%%%BILD ENDE

\subsection{Unterschied bei kartesischer Darstellung des Windes}
\highlight{Vergleich kartesisch und polar am besten Modell}

\subsection{Unterschied bei Einbeziehung weiterer Messstationen}

\highlight{Was passiert, wenn die umliegenden Stationen mit einbezogen werden und was wenn nicht? Sind Unterschiede zu sehen?}

\subsection{Vergleich mit Benchmarkmodell}

\highlight{
	\begin{itemize}
		\item Vergleich von MAE, RMSE, MAPE, etc.
		\item Vergleich schwer, da Datenbasis nicht gleich
	\end{itemize}
}

\section{Weiterentwicklung des Modells \highlight{Ausblick}}

\highlight{
Ideen, mit dem man das Modell / die Ergebnisse verbassern kann:
\begin{enumerate}
	\item Andere Stationsdaten mit weniger volatier Windgeschwindigkeit (d.h. Daten am Meer)
	\item Ensemble-Modell
	\item Convolution hinzufügen, wie z.B. \cite{2019_Chen_MultifactorSpatiotemporalCorrelation}
	\item Satellitendaten nutzen
	\item Zeitlich geringere Auflösung nehmen, weil wir in Ulm relativ instationäre und leichte Winde haben
	\item MFSTC Modell für mehrere Standorte
	\item SARIMA Modell kann unsere Ergebnisse besser in Kontext setzen: Es kann sein, dass Datenlage in Süddeutschland auch für herkömmliches Modell schwer vorherzusagen ist.
\end{enumerate}
}
%%%%%%%%%%%%%%%%%%%%%%%%%%%%%%%%%%%%%%%%%%%%%%%%%%%%%%%%%%%%%%%%%%%%%%%
%                                                                     %
%                                                                     %
%                                                                     %
%                               KAPITEL                               %
%                                                                     %
%                                                                     %
%                                                                     %
%%%%%%%%%%%%%%%%%%%%%%%%%%%%%%%%%%%%%%%%%%%%%%%%%%%%%%%%%%%%%%%%%%%%%%%
% \chapter{Test und Wissen}

% \section{Tabelle}
% %%%%%TABELLE ANFANG
% \begin{table}[ht]
% 	\centering
% 	\caption{Das hier ist eine Testtabelle, man beachte die gezwungene Breite in der rechten Spalte. Lässt sich einfach durch den Befehl C\{5cm\} erzeugen.}
% 	\begin{tabular}{|l|C{5cm}|}
% 		\hline
%         \rowcolor{color80}
% 		\textbf{Erste Zelle}&\textbf{Ein Header}\\
% 		\hline
% 		Moin: & Zusammen\\\hline
% 		leer:&\\\hline
% 		Moin: & Zusammen\\\hline
% 		leer:&\\\hline
% 	\end{tabular}
% \label{tab:testtabelle}
% \end{table}
% %%%%%TABELLE ENDE

% %%%%%LANGTABELLE ANFANG
% \begin{longtable}{|L{3.6cm}|L{6cm}|L{6cm}|}
% 	\caption{Finale Merkmale}\label{tab:longtable}\\
%     % Definition des Tabellenkopfes auf der ersten Seite
% 	\hline
%     \rowcolor{color80}
% 	\textbf{Abkürzung}&\textbf{Englisch}&\textbf{Deutsch}\\
% 	\hline
% 	\endfirsthead % Erster Kopf zu Ende
% 	% Definition des Tabellenkopfes auf den folgenden Seiten
% 	\hline
% 	\rowcolor{color80}
% 	\textbf{Abkürzung}&\textbf{Englisch}&\textbf{Deutsch}\\
% 	\hline
% 	\endhead % Zweiter Kopf ist zu Ende
%     \hline
%     \endfoot
%     \hline
%     \endlastfoot
% 	% Ab hier kommt der Inhalt der Tabelle
% 	$test_{\mathrm{abk}}$&ein sehr sehr sehr langertextmit langen wörternein sehr sehr sehr langertextmit langen wörternein sehr sehr sehr langertextmit langen wörternein sehr sehr sehr langertextmit langen wörternein sehr sehr sehr langertextmit langen wörtern&Einzeilig\\
% 	Z&5&as\\\hline
% 	A&1&91\\\hline
% 	B&2&97\\\hline
% 	Z&5&as\\\hline
% 	A&1&91\\\hline
% 	B&2&97\\\hline
% 	A&1&91\\\hline
% 	B&2&97\\\hline
% 	Z&5&as\\\hline
% 	A&1&91\\\hline
% 	B&2&97\\\hline
%     Z&5&as\\\hline
% 	A&1&91\\\hline
% 	B&2&97\\\hline
% 	Z&5&as\\\hline
% 	A&1&91\\\hline
% 	B&2&97\\\hline
%     Z&5&as\\\hline
% 	A&1&91\\\hline
% 	B&2&97\\\hline
% 	Z&5&as\\\hline
% 	A&1&91\\\hline
% 	B&2&97\\\hline
%     Z&5&as\\\hline
% 	A&1&91\\\hline
% 	B&2&97\\\hline
% 	Z&5&as\\\hline
% 	A&1&91\\\hline
% 	B&2&97\\\hline
%     Z&5&as\\\hline
% 	A&1&91\\\hline
% 	B&2&97\\\hline
% 	Z&5&as\\\hline
% 	A&1&91\\\hline
% 	B&2&97\\\hline
%     Z&5&as\\\hline
% 	A&1&91\\\hline
% 	B&2&97\\\hline
% 	Z&5&as\\\hline
% 	A&1&91\\\hline
% 	B&2&97\\\hline
% 	Z&5&as\\\hline
% 	A&1&91\\\hline
% 	B&2&97\\\hline
% 	Z&5&as\\\hline
% 	A&1&91\\\hline
% 	B&2&97\\\hline
% 	Z&5&KEINE hline AM SCHLUSS!!!\\
% \end{longtable}
% %%%%%LANGTABELLE ENDE

% \section{Bild}

% %%%%%BILD ANFANG
% \begin{figure}[tph]
% 	\begin{center}
% 		\includegraphics[scale = 1]{./images/0_testbild.png}
% 		\caption{TEST Hier ist ein wunderschönes Testbild zu erkennen, und das hier ist ein wunderschönes Testbild zu erkennen, und das hier ist ein wunderschönes Testbild zu erkennen, und das hier ist ein wunderschönes Testbild zu erkennen, und das hier ist ein wunderschönes Testbild zu erkennen, und das hier ist ein wunderschönes Testbild zu erkennen, und das hier ist ein}
% 		\label{fig:testbild}
% 	\end{center}
% \end{figure}
% %%%%%BILD ENDE

% %%%%%DOPPELBILD ANFANG
% \begin{figure}
% 	\centering
% 	\begin{minipage}[b]{.4\linewidth} % [b] => Ausrichtung an \caption
% 		\includegraphics[width=\linewidth]{./images/0_testbild.png}
% 		\caption{Beschreibung und so}
% 		\label{fig:testbild_klein_links}
% 	\end{minipage}
% 	\hspace{.1\linewidth}% Abstand zwischen Bilder
% 	\begin{minipage}[b]{.4\linewidth} % [b] => Ausrichtung an \caption
% 		\includegraphics[width=\linewidth]{./images/0_testbild.png}
% 		\caption{Zweite Beschreibung}
% 		\label{fig:testbild_klein_rechts}
% 	\end{minipage}
% \caption*{Hier steht eine etwas längere Bildunterschrift, die dieses Doppelbild beschreibt. Zu sehen sind weder Land noch Wasser, das macht bei Testbildern aber nix.}
% \end{figure}
% %%%%%DOPPELBILD ENDE

% \section{Formeln}

% Aufpassen, hier kommt jetzt eine einfache Testformel hin:

% \begin{equation}
% \label{eq:1}
% x=10
% \end{equation}

% Diese Formel \autoref{eq:1} kann auch referenziert werden.

% \begin{equation}
% x=Ma_{\mathrm{blabla}} \cdot y
% \end{equation}

% \section{Zitate}

% Um Bib zu kompilieren, einmal F8 drücken.

% Bild \autoref{fig:testbild} referenzieren.
% %Auch ein Autorname \citeauthor{ernstGrundkursInformatikGrundlagen2015} oder ein Jahr geht \citeyear{ernstGrundkursInformatikGrundlagen2015}

%%%%%%%%%%%%%%%%%%%%%%%%%%%%%%%%%%%%%%%%%%%%%%%%%%%%%%%%%%%%%%%%%%%%%%%
%                                                                     %
%                                                                     %
%                                                                     %
%                               KAPITEL                               %
%                                                                     %
%                                                                     %
%                                                                     %
%%%%%%%%%%%%%%%%%%%%%%%%%%%%%%%%%%%%%%%%%%%%%%%%%%%%%%%%%%%%%%%%%%%%%%%
\printbibliography[title=Literaturverzeichnis]
\end{document}